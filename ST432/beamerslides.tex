\documentclass[notes=show]{beamer}
\setbeamertemplate{footline}[page number]
\useinnertheme{circles}
\beamertemplatenavigationsymbolsempty
\hypersetup{colorlinks=true,linkcolor=blue} 
\usefonttheme[stillsansseriftext]{serif}
\usepackage{multicol}
\usepackage{fancyvrb}
\newcommand\nm[1]{\left\lVert#1\right\rVert}
\usetheme{CambridgeUS}
\setbeamercolor{frametitle}{,bg=brown!50}
\usepackage{ biblatex }

\newcommand{\BF}[1]{{\bf #1:}\hspace{1em}\ignorespaces}
\newcommand{\bg}[1]{\mbox{{\boldmath ${#1}$}}}

\newcommand \bit  {\begin{itemize}}
\newcommand \eit  {\end{itemize}}
\newcommand \ben  {\begin{enumerate}}
\newcommand \een  {\end{enumerate}}

\newcommand{\bx}{\mathbf{x}}
\newcommand{\x}{\mathrm{x}}
\newcommand{\bX}{\mathbf{X}}
\newcommand{\X}{\mathrm{X}}

\newcommand{\textR}[1]{\textcolor{blue}{\texttt{#1}}}
\newcommand{\R}{\textR{R}}

\definecolor{WPICrim}{RGB}{172, 43, 55}
\setbeamercolor{title}{fg=WPICrim}
\setbeamercolor{palette tertiary}{bg=WPICrim}
\setbeamercolor{palette primary}{fg=WPICrim}
\setbeamercolor{structure}{fg=WPICrim}

\title[RPCA]{\textbf{An Overview of PCA \& RPCA with Implementation } }
\author[Applied Statistics Group\setbeamercolor{frametitle}{fg=Brown,bg=Brown!20}
\setbeamercolor{section in head/foot}{bg=Brown}
\setbeamercolor{author in head/foot}{bg=Brown}
\setbeamercolor{date in head/foot}{fg=Brown}]{N.Gawron \and G.Johnson \and A.Sibu \and A.Stapelton }
\institute[WPI]{WPI Mathematical Sciences Applied Statistics}
\date[Summer 2021]{Summer 2021}
\date{Monday June 14th, 2021}

\begin{document}

\frame{\titlepage}


 


\frame{ \frametitle{Future Objectives \& Applications}
 \vspace{-2mm}
\bit
\item Absolute Return Bond and Currencies (ARBC) Group uses PCA on their investment data
\item Correlation between variables breaks down, the linearity of the PCA hinders the detection of this break in a timely manner
\item Our group will provide research to investigate AI / ML framework using techniques on the data set provided by ARBC to detect these decorrelation events
\item RPCA,  Autoencoders and more methods will be studied and utilized
\item This week PCA and RPCA were examined

\eit

}


\frame{ \frametitle{PCA / Principal Component Analysis}
 \vspace{-2mm}
A useful statistical tool which applies various linear algebra techniques.
 \vspace{1mm}
\newline {General  Goals:} 
{\small 
\ben
\item  to eliminate redundancy by taking account of correlation among variables 
\item to reduce the dimension of a dataset by building latent variables that contain most of the information.
\item to create an orthogonal basis in which to view our data that explains large amounts of variability
\ben
\item Orthogonal basis vectors are the Principal Components (PCs ) \& are Linear Combinations (LCs) of variables
\item PCs maximizing the variance
\een
\item to pre-process data  \& use PCs to prevent overfitting of models
\een
 
\vspace{-2mm}

}}



\frame{ \frametitle{Backbone of PCA }
\vspace{-2mm}
Application of Singular Value Decompostion (SVD)
\bit
\item Data: $\{x_1, \ldots, x_n\} \in \mathbb{R}^p$;  \# of features is $p$.\footnote{WLOG, all vectors are column vectors with \textbf{numeric} values.}

\item  Data matrix: $\X_{n\times p}$ = $(x_1, \ldots, x_n)'$.
\item  Centered data matrix: $\tilde \X$ = $(\tilde x_1, \ldots, \tilde x_n)'$ where $\tilde x_j$  = $x_j - \bar x$.
\item SVD: $\tilde\X = U_{n\times p}D_{p\times p} V_{p\times p}'$ where 
\bit
\item Orthogonal by Definition:  $U'U = V'V = I_p$
\item $D = diag(\sigma_1, \sigma_2, \ldots, \sigma_p)$, with $\sigma_1\ge\sigma_2 \ge\ldots  \ge \sigma_p\geq0$.  
\eit

\item $\tilde\X VV' = UD V' = \tilde\X$.
\eit

 
Use $\tilde\X V_kV_k'$ to approximate $\tilde\X$, where $V_k$ is the first $k$ columns of $V$. 
}
 


\begin{frame}[fragile]{PCA in \R \, Iris Summary Data}


\begin{columns}[onlytextwidth,b]
      \column{\dimexpr\linewidth-50mm-5mm}
{\tiny
\begin{Verbatim}[formatcom=\color{blue}]
> data(iris)
> head(iris)
  Sepal.Length Sepal.Width Petal.Length Petal.Width Species
1          5.1         3.5          1.4         0.2  setosa
2          4.9         3.0          1.4         0.2  setosa
3          4.7         3.2          1.3         0.2  setosa
4          4.6         3.1          1.5         0.2  setosa
5          5.0         3.6          1.4         0.2  setosa
6          5.4         3.9          1.7         0.4  setosa

> iris_stats
  measurement  count  mean median    sd
1 Petal.Length   150  3.76   4.35 1.77 
2 Petal.Width    150  1.20   1.3  0.762
3 Sepal.Length   150  5.84   5.8  0.828
4 Sepal.Width    150  3.06   3    0.436
\end{Verbatim}
}
   

      \column{50mm}
      
\includegraphics[scale =.36]{Iris_Variables.jpg}
    \end{columns}

\end{frame}


\begin{frame}[fragile]{PCA in \R \, on IRIS Data}
\bit
\item Pairs Plot on Iris
\eit
\includegraphics[scale = .35]{PairsonIris.jpeg}

\end{frame}




\begin{frame}[fragile]{PCA in \R \, General Outputs}

{\tiny
\begin{Verbatim}[formatcom=\color{blue}]
> M <- as.matrix(iris[,1:4])  # 150*4
> pr.out <- prcomp(M, center = TRUE) # Try "scale = TRUE"
> (pr.var = pr.out$sdev^2)
[1] 4.22824171 0.24267075 0.07820950 0.02383509

> pr.out$rotation
                     PC1         PC2         PC3        PC4
Sepal.Length  0.36138659 -0.65658877  0.58202985  0.3154872
Sepal.Width  -0.08452251 -0.73016143 -0.59791083 -0.3197231
Petal.Length  0.85667061  0.17337266 -0.07623608 -0.4798390
Petal.Width   0.35828920  0.07548102 -0.54583143  0.7536574

par(mfrow = c(1,2))
barplot(pr.out$rotation[,1],main="PC1 loadings")
barplot(pr.out$rotation[,2],main="PC2 loadings")
\end{Verbatim}
}

\includegraphics[scale=.23]{pca_loadings}
\end{frame}


\begin{frame}[fragile]{PCA in \R \, Visualization}

{\tiny
\begin{Verbatim}[formatcom=\color{blue}]
par(mfrow = c(1,1))
plot(pr.out$x[,1:2],col=iris[,5])
\end{Verbatim}
}

\includegraphics[scale=.25]{pca_scores}



\end{frame}






\begin{frame}{How many components to keep?}

\ben
\item Kaiser's rule (of thumb): Retain PCs 1 thru $k$ satisfying 
$$\lambda_k > \frac{1}{p} \sum_{j=1}^p \lambda_j $$ 
Tends to choose fewer components.
\item Find $k_0$ that minimizes 
$$ \frac{\lambda_{k+1}}{\lambda_k}, \quad \quad k = 1, 2, \ldots, p-1. $$ 
\een

\end{frame}


\begin{frame}[fragile]{PCA in \R \,with Scree}

{\tiny
\begin{Verbatim}[formatcom=\color{blue}]
par(mfrow = c(1,2))
plot(pr.var, xlab="Principal Component",ylab="Eigenvalue",type='b')
pve = pr.var/sum(pr.var)
plot(cumsum(pve), xlab="Principal Component",
     ylab=" Cumulative Proportion of Variance Explained ", ylim=c(0,1), type='b')
abline(h = .9, lty = 2)
\end{Verbatim}
}

 
\includegraphics[scale=.25]{pca_scree}
\end{frame}








\frame{
\frametitle{Goals of RPCA}

{\small 
\bit
\item Robust to measurement anomalies or outliers
\item Remove anomalies and provide a low rank approximation of the original data

\eit

\begin{equation}
\tilde \X_{n\times p} = L_{n\times p} + S_{n\times p}
\end{equation}
where $L$ = low rank; $S$ = sparse (anomalies)
\ben
\item  Use $L$ to approximate $\tilde \X$.
\een
}

}








\frame{
\frametitle{RPCA Backbone}

{\small 
\begin{equation}
\tilde\X_{n\times p} = L_{n\times p} + S_{n\times p}
\end{equation}




\ben
\item How to obtain $L$ and $S$? 
\begin{equation}
\min_{L,S} r(L) + \lambda \|S\|_0 \quad \text{, subject to }\tilde\X = L+S.
\end{equation} 
\bit
\item $r(L)$= rank of $L$; $\nm{S}_0$ = \# of non-zero entries in $S$
\item computationally expensive if there are many anomalies. 
\eit

\item Convex optimization: 
\begin{equation}
\min_{L,S} \| L\|_{*} + \lambda \|S\|_1, \quad \text{subject to } \tilde\X = L+S
\end{equation}
\vspace{-3mm}
\bit
\item $\| L\|_{*}$ = nuclear norm of $L$ (sum of SVs of $L$) 
\item $\|S\|_1$ = $\sum_{i,j} |S_{i,j}|$
\item $\lambda$ = $\frac{1}{\sqrt{\max(n,p)}}$ 
\item Principal Component Pursuit / Augmented Lagrangian method
\eit

\een


}

}





\begin{frame}[fragile]{\R \, code Implementation on Iris}

{\tiny
\begin{Verbatim}[formatcom=\color{blue}]
library(rpca)
Mcent <- sweep(M,2,colMeans(M)) # center data matrix
res <- rpca(Mcent)
> names(res)
[1] "L"           "S"           "L.svd"       "convergence"
> res$L.svd$d  # The low rank component has rank 2
[1] 20.060305  1.604899
>
> mean(res$S==0) # sparse component is not sparse
[1] 0.2716667

## Plot the first (the only) 2 principal components of  L
rpc <- res$L.svd$u%*%diag(res$L.svd$d)
plot(rpc,col=iris[,5])
\end{Verbatim}
}
\vspace{-1mm}
\includegraphics[scale=.25]{rpca_scatter .png}
%  \textbf{convergence\$converged
% TRUE if the algorithm converged with respect to term.delta.
% convergence\$iterations
% Number of performed iterations.
% convergence\$final.delta
% The final iteration delta which is compared with term.delta.
% convergence\$all.delta
% All delta from all iterations.}

\end{frame}




\begin{frame}[fragile]{\R \, code Plots}

{\tiny
\begin{Verbatim}[formatcom=\color{blue}]
plot(pr.out$x[,1:2],col=iris[,5]) # PCA
points(rpc,col=iris[,5],pch="+")  # RPCA
\end{Verbatim}

\includegraphics[scale=.35]{rpca_pca}

}

 
\end{frame}



\begin{frame}[fragile]{\R \, code for Low Rank Approximation Plots}

{\tiny
\begin{Verbatim}[formatcom=\color{blue}]
## Plot measurements against measurements corrected by sparse components
par(mfcol=c(2,2))
for(i in 1:4) {
  plot(M[,i],M[,i]-res$S[,i],col=iris[,5],xlab=colnames(M)[i])}
\end{Verbatim}

\includegraphics[scale=.36]{rpca_corrected}

}

 
\end{frame}

\begin{frame}[fragile]{Introduction of Wine Data}

\bit
\item Wine Quality and Metrics UCI Machine Learning Repository
\eit
\vspace{.25cm}
{\tiny
\begin{Verbatim}[formatcom=\color{blue}]
colnames(wine) <- c("Cvs","Alcohol","MalicAcid","Ash","AlcalinityOfAsh", "Magnesium", 
"TotalPhenols", "Flavanoids", "NonflavanoidPhenols", "Proanthocyanins", "ColorIntensity", "Hue", 
"ODRatio", "Proline")
# The first column corresponds to the classes
wineClasses <- factor(wine$Cvs)


\end{Verbatim}
}
\vspace{-.5cm}
\hspace{-.5cm}
\includegraphics[scale = .30]{wine_boxplots.png}

 
\end{frame}

\begin{frame}[fragile]{Summary Statistics of Wine Data}

\begin{Verbatim}[formatcom=\color{blue}]
                    count        mean  median          sd
Alcohol               178  13.0006180  13.050   0.8118265
MalicAcid             178   2.3363483   1.865   1.1171461
Ash                   178   2.3665169   2.360   0.2743440
AlcalinityOfAsh       178  19.4949438  19.500   3.3395638
Magnesium             178  99.7415730  98.000  14.2824835
TotalPhenols          178   2.2951124   2.355   0.6258510
Flavanoids            178   2.0292697   2.135   0.9988587
NonflavanoidPhenols   178   0.3618539   0.340   0.1244533
Proanthocyanins       178   1.5908989   1.555   0.5723589
ColorIntensity        178   5.0580899   4.690   2.3182859
Hue                   178   0.9574494   0.965   0.2285716
ODRatio               178   2.6116854   2.780   0.7099904
Proline               178 746.8932584 673.500 314.9074743
    
\end{Verbatim}
    
\end{frame}

\begin{frame}[fragile]{Correlation of Wine Data \R}

\hspace{-.5cm}
\includegraphics[scale = .34]{wine_scatterplot.png}

\end{frame}



\begin{frame}[fragile]{\R \, code for Outlier Control (Wine)}

\bit
\item General PCA on Wine Data
\eit
\vspace{.25cm}
{\tiny
\begin{Verbatim}[formatcom=\color{blue}]

winePCA <- prcomp(scale(wine[,-1]))
#plots all 178 observations, but using just the 1st & 2nd PCs. Color coded by cultivator
plot(winePCA$x[,1:2], col = wineClasses)
\end{Verbatim}


\includegraphics[scale = .40]{wine_pca.png}
}



 
\end{frame}



\begin{frame}[fragile]{\R \, code for PC Selection Methods}

\bit
\item PCA diagnostics / Kaiser’s 
\eit
\vspace{.25 cm}
{\tiny
\begin{Verbatim}[formatcom=\color{blue}]
    summary(winePCA) #8 PCs required to explain at least 90% of variance
Importance of components:
                         PC1    PC2    PC3     PC4     PC5     PC6     PC7
Standard deviation     2.169 1.5802 1.2025 0.95863 0.92370 0.80103 0.74231
Proportion of Variance 0.362 0.1921 0.1112 0.07069 0.06563 0.04936 0.04239
Cumulative Proportion  0.362 0.5541 0.6653 0.73599 0.80162 0.85098 0.89337
                           PC8     PC9   PC10    PC11    PC12    PC13
Standard deviation     0.59034 0.53748 0.5009 0.47517 0.41082 0.32152
Proportion of Variance 0.02681 0.02222 0.0193 0.01737 0.01298 0.00795
Cumulative Proportion  0.92018 0.94240 0.9617 0.97907 0.99205 1.00000


> winePCA$sdev
 [1] 2.1692972 1.5801816 1.2025273 0.9586313 0.9237035 0.8010350 0.7423128
 [8] 0.5903367 0.5374755 0.5009017 0.4751722 0.4108165 0.3215244
> winePCA$sdev^2
 [1] 4.7058503 2.4969737 1.4460720 0.9189739 0.8532282 0.6416570 0.5510283
 [8] 0.3484974 0.2888799 0.2509025 0.2257886 0.1687702 0.1033779
> mean(winePCA$sdev^2)
[1] 1

eigenvalue ratio minimization
> for(i in 1:12) { wine_ratio[i] = wine_eigen[i+1] / wine_eigen[i] }
> cbind(which(wine_ratio==min(wine_ratio)),min(wine_ratio))
     [,1]           [,2]
[1,]    1 0.    5306105
\end{Verbatim}
}
\end{frame}




\begin{frame}[fragile]{Wine Data \R \, code Plot Interpretations}

\bit
\item PCA diagnostics
\eit
\vspace{.25 cm}
{\tiny
\begin{Verbatim}[formatcom=\color{blue}]
PercVar <- winePCA$sdev^2 / sum(winePCA$sdev^2)
> par(mfcol=c(1,2))
> periodT<-rep("o",178)
> biplot(winePCA,xlabs= periodT)
> plot(cumsum(PercVar), xlab="Principal Component",
ylab=" Cumulative Proportion of Variance ", ylim=c(0,1))
> abline(h=.8,lty=2)
\end{Verbatim}

\vspace{-1mm}
\includegraphics[scale=.31]{NewBiPlotandScree.jpeg}
     
}
\end{frame}



\begin{frame}[fragile]{Wine Data \R \, code Plot Interpretations}

\bit
\item Biplots on Other PCs
\eit
\vspace{.25 cm}
{\tiny
\begin{Verbatim}[formatcom=\color{blue}]
PercVar <- winePCA$sdev^2 / sum(winePCA$sdev^2)
> par(mfcol=c(1,2))
> biplot(winePCA, choices = 2:3, xlabs= periodT)
> biplot(winePCA, choices = c(1,3), xlabs= periodT)
\end{Verbatim}

\vspace{-.2mm}
\includegraphics[scale=.45]{pc3.png}
     
}
\end{frame}


\begin{frame}[fragile]{\R \, code for Outlier Control}

\bit
\item PCA is non-resistant to outliers
\eit
\vspace{.25 cm}
{\tiny
\begin{Verbatim}[formatcom=\color{blue}]
#Make an outlier 
wineOutlier <- wine
wineOutlier[10,] <- wineOutlier[10,]*10 
# change the 10th obs. into an extreme one by multiplying its profile by 10
outlierPCA <- prcomp(scale(wineOutlier[,-1]))
plot(outlierPCA$x[,1:2], col = wineClasses)
#outlier sample is very obvious from this graph 
\end{Verbatim}

\includegraphics[scale=.40]{wine_outlier_pca.png}

}
\end{frame}

\begin{frame}[fragile]{\R \, code for RPCA Outlier Control}

\bit
\item RPCA is resistant to outliers and scales
\eit
\vspace{.25 cm}
{\tiny
\begin{Verbatim}[formatcom=\color{blue}]

#using rpca on wine data 
wineRobust = rpca(scale(wineOutlier[-1]))
wineRobust$L.svd$d #the low rank component has rank 8 
wineRobustPCs = wineRobust$L.svd$u%*%diag(wineRobust$L.svd$d)
mean(wineRobust$S==0) #19.7% of the sparse matrix are zeros 
plot(wineRobustPCs, col = wineClasses)
\end{Verbatim}

\includegraphics[scale=.40]{wine_robust_pca.png}

}


\end{frame}

\begin{frame}[fragile]{Viability of RPCA in \R \, on Wine Data }
\vspace{-2mm}
{\tiny
\begin{Verbatim}[formatcom=\color{blue}]
M<-as.matrix(wine[,-1])
res <- rpca(sweep(M,2,colMeans(M)))
res$L.svd$d #the low rank component has rank 8
resPCs <- res$L.svd$u%*%diag(res$L.svd$d)

par(mfcol=c(4,4))
for(i in 1:13) {
  plot(M[,i],M[,i]-res$S[,i],col=iris[,5],xlab=colnames(M)[i])
\end{Verbatim}
}

\end{frame}


\begin{frame}{Plotting Values}
    \includegraphics[scale =.45]{ResPlotApprox.jpeg}
\end{frame}



\frame{
\frametitle{Sparse PCA}
A  disadvantage of general PCA is that the PCs are usually linear combinations of \underline{all} input variables. 
{ \small
\bit
\item Sparse PCA finds certain columns from data matrix $M$ to use select input variables
\item Loss of Orthogonality in PCs
\eit

}


}





\frame{
\frametitle{Sparse PCA Backbone}
SPCA assumes we can reconstruct a matrix $X$ of rank $k$ using any $k$ columns of $X$ 
{ \small
\bit
\item center the data matrix to create $\tilde X$
\item create a selection matrix $\Phi_k\in \{0,1\}^{n\times k}$ such that $\tilde X_k = \tilde X\Phi_k$
\item perform PCA on $\tilde X_k$ to achieve $U\Sigma V^T = SVD(\tilde X_k)$
\item compute $Z = \tilde X_k(\Phi_k^T-\Phi_k^TVV^T)((I-\Phi_k\Phi_k^T)VV^T-(I-\Phi_k\Phi_k^T))^{-1}$
\eit
\ben
\item use $L = X_k\Phi_k\Phi_k^T+Z(I-\Phi_k\Phi_k^T)$ to approximate $X$
\een

}

}





\begin{frame}[fragile]{Sparse PCA in \R \, }

{\tiny
\begin{Verbatim}[formatcom=\color{blue}]
> library('sparsepca')
> spca_wine = spca(wine,k=7,scale = TRUE)
> wineClasses <- factor(wine$Cvs)
> plot(spca_wine$scores[,1:2], col = > wineClasses,xlab = 'PC1',ylab = 'PC2')
\end{Verbatim}
}

\includegraphics[scale=.30]{spca1.png}

\end{frame}




\begin{frame}[fragile]{Thanks!}

\begin{center}
  Questions?
\end{center}

\end{frame}





\end{document}






