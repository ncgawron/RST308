% Options for packages loaded elsewhere
\PassOptionsToPackage{unicode}{hyperref}
\PassOptionsToPackage{hyphens}{url}
%
\documentclass[
]{article}
\usepackage{lmodern}
\usepackage{amssymb,amsmath}
\usepackage{ifxetex,ifluatex}
\ifnum 0\ifxetex 1\fi\ifluatex 1\fi=0 % if pdftex
  \usepackage[T1]{fontenc}
  \usepackage[utf8]{inputenc}
  \usepackage{textcomp} % provide euro and other symbols
\else % if luatex or xetex
  \usepackage{unicode-math}
  \defaultfontfeatures{Scale=MatchLowercase}
  \defaultfontfeatures[\rmfamily]{Ligatures=TeX,Scale=1}
\fi
% Use upquote if available, for straight quotes in verbatim environments
\IfFileExists{upquote.sty}{\usepackage{upquote}}{}
\IfFileExists{microtype.sty}{% use microtype if available
  \usepackage[]{microtype}
  \UseMicrotypeSet[protrusion]{basicmath} % disable protrusion for tt fonts
}{}
\makeatletter
\@ifundefined{KOMAClassName}{% if non-KOMA class
  \IfFileExists{parskip.sty}{%
    \usepackage{parskip}
  }{% else
    \setlength{\parindent}{0pt}
    \setlength{\parskip}{6pt plus 2pt minus 1pt}}
}{% if KOMA class
  \KOMAoptions{parskip=half}}
\makeatother
\usepackage{xcolor}
\IfFileExists{xurl.sty}{\usepackage{xurl}}{} % add URL line breaks if available
\IfFileExists{bookmark.sty}{\usepackage{bookmark}}{\usepackage{hyperref}}
\hypersetup{
  pdftitle={Gawron\_Homework6},
  pdfauthor={Nicholas Gawron},
  hidelinks,
  pdfcreator={LaTeX via pandoc}}
\urlstyle{same} % disable monospaced font for URLs
\usepackage[margin=1in]{geometry}
\usepackage{color}
\usepackage{fancyvrb}
\newcommand{\VerbBar}{|}
\newcommand{\VERB}{\Verb[commandchars=\\\{\}]}
\DefineVerbatimEnvironment{Highlighting}{Verbatim}{commandchars=\\\{\}}
% Add ',fontsize=\small' for more characters per line
\usepackage{framed}
\definecolor{shadecolor}{RGB}{248,248,248}
\newenvironment{Shaded}{\begin{snugshade}}{\end{snugshade}}
\newcommand{\AlertTok}[1]{\textcolor[rgb]{0.94,0.16,0.16}{#1}}
\newcommand{\AnnotationTok}[1]{\textcolor[rgb]{0.56,0.35,0.01}{\textbf{\textit{#1}}}}
\newcommand{\AttributeTok}[1]{\textcolor[rgb]{0.77,0.63,0.00}{#1}}
\newcommand{\BaseNTok}[1]{\textcolor[rgb]{0.00,0.00,0.81}{#1}}
\newcommand{\BuiltInTok}[1]{#1}
\newcommand{\CharTok}[1]{\textcolor[rgb]{0.31,0.60,0.02}{#1}}
\newcommand{\CommentTok}[1]{\textcolor[rgb]{0.56,0.35,0.01}{\textit{#1}}}
\newcommand{\CommentVarTok}[1]{\textcolor[rgb]{0.56,0.35,0.01}{\textbf{\textit{#1}}}}
\newcommand{\ConstantTok}[1]{\textcolor[rgb]{0.00,0.00,0.00}{#1}}
\newcommand{\ControlFlowTok}[1]{\textcolor[rgb]{0.13,0.29,0.53}{\textbf{#1}}}
\newcommand{\DataTypeTok}[1]{\textcolor[rgb]{0.13,0.29,0.53}{#1}}
\newcommand{\DecValTok}[1]{\textcolor[rgb]{0.00,0.00,0.81}{#1}}
\newcommand{\DocumentationTok}[1]{\textcolor[rgb]{0.56,0.35,0.01}{\textbf{\textit{#1}}}}
\newcommand{\ErrorTok}[1]{\textcolor[rgb]{0.64,0.00,0.00}{\textbf{#1}}}
\newcommand{\ExtensionTok}[1]{#1}
\newcommand{\FloatTok}[1]{\textcolor[rgb]{0.00,0.00,0.81}{#1}}
\newcommand{\FunctionTok}[1]{\textcolor[rgb]{0.00,0.00,0.00}{#1}}
\newcommand{\ImportTok}[1]{#1}
\newcommand{\InformationTok}[1]{\textcolor[rgb]{0.56,0.35,0.01}{\textbf{\textit{#1}}}}
\newcommand{\KeywordTok}[1]{\textcolor[rgb]{0.13,0.29,0.53}{\textbf{#1}}}
\newcommand{\NormalTok}[1]{#1}
\newcommand{\OperatorTok}[1]{\textcolor[rgb]{0.81,0.36,0.00}{\textbf{#1}}}
\newcommand{\OtherTok}[1]{\textcolor[rgb]{0.56,0.35,0.01}{#1}}
\newcommand{\PreprocessorTok}[1]{\textcolor[rgb]{0.56,0.35,0.01}{\textit{#1}}}
\newcommand{\RegionMarkerTok}[1]{#1}
\newcommand{\SpecialCharTok}[1]{\textcolor[rgb]{0.00,0.00,0.00}{#1}}
\newcommand{\SpecialStringTok}[1]{\textcolor[rgb]{0.31,0.60,0.02}{#1}}
\newcommand{\StringTok}[1]{\textcolor[rgb]{0.31,0.60,0.02}{#1}}
\newcommand{\VariableTok}[1]{\textcolor[rgb]{0.00,0.00,0.00}{#1}}
\newcommand{\VerbatimStringTok}[1]{\textcolor[rgb]{0.31,0.60,0.02}{#1}}
\newcommand{\WarningTok}[1]{\textcolor[rgb]{0.56,0.35,0.01}{\textbf{\textit{#1}}}}
\usepackage{graphicx,grffile}
\makeatletter
\def\maxwidth{\ifdim\Gin@nat@width>\linewidth\linewidth\else\Gin@nat@width\fi}
\def\maxheight{\ifdim\Gin@nat@height>\textheight\textheight\else\Gin@nat@height\fi}
\makeatother
% Scale images if necessary, so that they will not overflow the page
% margins by default, and it is still possible to overwrite the defaults
% using explicit options in \includegraphics[width, height, ...]{}
\setkeys{Gin}{width=\maxwidth,height=\maxheight,keepaspectratio}
% Set default figure placement to htbp
\makeatletter
\def\fps@figure{htbp}
\makeatother
\setlength{\emergencystretch}{3em} % prevent overfull lines
\providecommand{\tightlist}{%
  \setlength{\itemsep}{0pt}\setlength{\parskip}{0pt}}
\setcounter{secnumdepth}{-\maxdimen} % remove section numbering

\title{Gawron\_Homework6}
\author{Nicholas Gawron}
\date{3/3/2021}

\begin{document}
\maketitle

\begin{Shaded}
\begin{Highlighting}[]
\NormalTok{knitr}\OperatorTok{::}\NormalTok{opts_chunk}\OperatorTok{$}\KeywordTok{set}\NormalTok{(}\DataTypeTok{echo =} \OtherTok{TRUE}\NormalTok{)}
\KeywordTok{library}\NormalTok{(tidyverse)}
\KeywordTok{library}\NormalTok{(readxl)}
\KeywordTok{library}\NormalTok{(tinytex)}
\end{Highlighting}
\end{Shaded}

Corresponding code for a problem's part will be \textbf{ABOVE} the
solution.

\hypertarget{problems}{%
\subsection{Problems}\label{problems}}

\begin{enumerate}
\def\labelenumi{\arabic{enumi}.}
\tightlist
\item
  A marketing research firm estimates the proportion of potential
  customers preferring a certain brand of lipstick by ``randomly''
  selecting 100 women who come by their booth in a shopping mall. Of the
  100 sampled, 65 women stated a preference for brand A.
\end{enumerate}

\begin{enumerate}
\def\labelenumi{\alph{enumi}.}
\item
  We would estimate the true proportion of women preferting brand A by
  calculating the proportion of women in the sample that had prefered
  the brand
\item
\item
\item
\end{enumerate}

\begin{enumerate}
\def\labelenumi{\arabic{enumi}.}
\setcounter{enumi}{1}
\tightlist
\item
  A school desires to estimate the average score that may be obtained on
  a reading comprehension exam for students in the sixth grade. The
  school's students are grouped into three tracks, with the faster
  learners in track I, the slower learners in track III, and the rest in
  track II. The school decides to stratify on tracks because this method
  should reduce the variability in test scores. The sixth grade contains
  55 students in track I, 80 in track II and 65 in track III. A
  stratified random sample of 50 students is proportionally allocated
  and yields simple random samples of \(n_1 = 14\), \(n_2 = 20\) and
  \(n_3=16\) from tracks I, II and III respectively. The data is
  contained in the accompanying excel file.
\end{enumerate}

\begin{enumerate}
\def\labelenumi{\alph{enumi}.}
\tightlist
\item
  We will consider the three groups
\end{enumerate}

\begin{Shaded}
\begin{Highlighting}[]
\NormalTok{ScoreData <-}\StringTok{ }\NormalTok{(readxl}\OperatorTok{::}\KeywordTok{read_excel}\NormalTok{(}\StringTok{'data/HW6Data.XLS'}\NormalTok{));ScoreData}
\end{Highlighting}
\end{Shaded}

\begin{verbatim}
## # A tibble: 20 x 3
##    `Track I` TrackII TrackIII
##        <dbl>   <dbl>    <dbl>
##  1        80      85       42
##  2        92      82       32
##  3        68      48       36
##  4        85      75       31
##  5        72      53       65
##  6        87      73       29
##  7        85      65       43
##  8        91      78       19
##  9        90      49       53
## 10        81      69       14
## 11        62      72       61
## 12        79      81       31
## 13        61      53       42
## 14        83      59       30
## 15        NA      68       39
## 16        NA      52       32
## 17        NA      71       NA
## 18        NA      61       NA
## 19        NA      59       NA
## 20        NA      42       NA
\end{verbatim}

\begin{Shaded}
\begin{Highlighting}[]
\NormalTok{Track1 <-}\StringTok{ }\NormalTok{(}\KeywordTok{na.omit}\NormalTok{(ScoreData}\OperatorTok{$}\StringTok{`}\DataTypeTok{Track I}\StringTok{`}\NormalTok{)); T1L <-}\StringTok{ }\KeywordTok{length}\NormalTok{(Track1)}
\NormalTok{Track2 <-}\StringTok{  }\NormalTok{(}\KeywordTok{na.omit}\NormalTok{(ScoreData}\OperatorTok{$}\NormalTok{TrackII)); T2L <-}\StringTok{ }\KeywordTok{length}\NormalTok{(Track2)}
\NormalTok{Track3 <-}\StringTok{  }\NormalTok{(}\KeywordTok{na.omit}\NormalTok{(ScoreData}\OperatorTok{$}\NormalTok{TrackIII)); T3L <-}\StringTok{ }\KeywordTok{length}\NormalTok{(Track3)}
\NormalTok{N <-}\StringTok{ }\NormalTok{T1L }\OperatorTok{+}\StringTok{ }\NormalTok{T2L }\OperatorTok{+}\StringTok{ }\NormalTok{T3L}

\NormalTok{Yst <-(}\DecValTok{1}\OperatorTok{/}\NormalTok{(N))}\OperatorTok{*}\NormalTok{((T1L}\OperatorTok{*}\KeywordTok{mean}\NormalTok{(Track1))}\OperatorTok{+}\NormalTok{(T2L}\OperatorTok{*}\KeywordTok{mean}\NormalTok{(Track2))}\OperatorTok{+}\NormalTok{(T3L}\OperatorTok{*}\KeywordTok{mean}\NormalTok{(Track3))); Yst}
\end{Highlighting}
\end{Shaded}

\begin{verbatim}
## [1] 60.2
\end{verbatim}

\begin{Shaded}
\begin{Highlighting}[]
\CommentTok{#FPC For each Strata}
\NormalTok{fpc1 <-}\StringTok{ }\NormalTok{(}\DecValTok{1}\OperatorTok{-}\NormalTok{(T1L}\OperatorTok{/}\NormalTok{N))}
\NormalTok{fpc2 <-}\StringTok{ }\NormalTok{(}\DecValTok{1}\OperatorTok{-}\NormalTok{(T2L}\OperatorTok{/}\NormalTok{N))}
\NormalTok{fpc3 <-}\StringTok{ }\NormalTok{(}\DecValTok{1}\OperatorTok{-}\NormalTok{(T3L}\OperatorTok{/}\NormalTok{N))}

\CommentTok{#variance }
\NormalTok{VYst =}\StringTok{ }\NormalTok{(}\DecValTok{1}\OperatorTok{/}\NormalTok{(N)}\OperatorTok{^}\DecValTok{2}\NormalTok{)}\OperatorTok{*}\NormalTok{(fpc1}\OperatorTok{*}\KeywordTok{sd}\NormalTok{(Track1)}\OperatorTok{^}\DecValTok{2}\OperatorTok{/}\NormalTok{T1L)}
\end{Highlighting}
\end{Shaded}

\begin{enumerate}
\def\labelenumi{\arabic{enumi}.}
\setcounter{enumi}{2}
\tightlist
\item
  We will now look at cavities.
\end{enumerate}

\begin{Shaded}
\begin{Highlighting}[]
\NormalTok{n <-}\DecValTok{10}\NormalTok{;N<-}\StringTok{ }\DecValTok{400}\NormalTok{;}
\NormalTok{Cavities <-}\StringTok{ }\KeywordTok{c}\NormalTok{(}\DecValTok{1}\NormalTok{ ,}\DecValTok{4}\NormalTok{ ,}\DecValTok{1}\NormalTok{ ,}\DecValTok{0}\NormalTok{ ,}\DecValTok{3}\NormalTok{ ,}\DecValTok{2}\NormalTok{, }\DecValTok{4}\NormalTok{ ,}\DecValTok{0}\NormalTok{ ,}\DecValTok{3}\NormalTok{ ,}\DecValTok{2}\NormalTok{)}
\end{Highlighting}
\end{Shaded}

\begin{enumerate}
\def\labelenumi{\arabic{enumi}.}
\setcounter{enumi}{3}
\tightlist
\item
  We will consider both parts
\end{enumerate}

\begin{Shaded}
\begin{Highlighting}[]
\NormalTok{Abv <-}\StringTok{  }\KeywordTok{c}\NormalTok{(}\DecValTok{48}\NormalTok{ ,}\FloatTok{48.7}\NormalTok{ ,}\FloatTok{50.1}\NormalTok{ ,}\FloatTok{43.3}\NormalTok{ ,}\FloatTok{47.5}\NormalTok{ ,}\FloatTok{49.4}\NormalTok{ ,}\FloatTok{39.9}\NormalTok{ ,}\DecValTok{52}\NormalTok{ ,}\FloatTok{46.7}\NormalTok{ ,}\FloatTok{50.5}\NormalTok{,}\FloatTok{45.6}\NormalTok{ ,}\FloatTok{49.7}\NormalTok{ ,}\FloatTok{45.3}\NormalTok{ ,}\FloatTok{46.9}\NormalTok{ ,}\FloatTok{48.5}\NormalTok{)}
\NormalTok{Bel <-}\StringTok{ }\KeywordTok{c}\NormalTok{(}\FloatTok{37.8}\NormalTok{, }\DecValTok{45}\NormalTok{, }\FloatTok{44.2}\NormalTok{, }\DecValTok{60}\NormalTok{, }\FloatTok{54.2}\NormalTok{, }\FloatTok{56.4}\NormalTok{, }\FloatTok{59.3}\NormalTok{, }\FloatTok{44.4}\NormalTok{, }\FloatTok{41.8}\NormalTok{, }\FloatTok{52.9}\NormalTok{,}\FloatTok{45.7}\NormalTok{, }\DecValTok{57}\NormalTok{, }\FloatTok{48.1}\NormalTok{, }\FloatTok{58.2}\NormalTok{, }\FloatTok{42.5}\NormalTok{, }\FloatTok{41.1}\NormalTok{)}
\NormalTok{AllTemp <-}\StringTok{ }\KeywordTok{c}\NormalTok{(Abv,Bel)}

\CommentTok{#part a}
\NormalTok{meanAll <-}\StringTok{ }\KeywordTok{mean}\NormalTok{(AllTemp ) }\CommentTok{# expected value of All temps }
\NormalTok{sdA<-}\StringTok{ }\KeywordTok{sd}\NormalTok{(AllTemp)}
\NormalTok{BdAll<-}\StringTok{ }\DecValTok{2}\OperatorTok{*}\KeywordTok{sqrt}\NormalTok{(sdA}\OperatorTok{^}\DecValTok{2}\OperatorTok{/}\KeywordTok{length}\NormalTok{(AllTemp)); }\CommentTok{# computes the Boundary}
\end{Highlighting}
\end{Shaded}

\begin{enumerate}
\def\labelenumi{\alph{enumi}.}
\tightlist
\item
  We will ignore the FPC for the simple random sample since the total
  population is all backyard pools which is trivially larger than 20
  times our smaple size of \(n=31\). The sample mean for an SRS is given
  by: 48.4096774. The standard deviation for the sample is calculated to
  be 5.7234229. The boundary, given by the formula:
  \(B = 2\sqrt{\frac{s^2}{n}}\) computes to 2.0559142. We are 95\%
  confident that the true mean backyard pool temperature is captured by
  the interval: \textbf{(46.3537632,50.4655916)}. Since
  \(50\in (46.3537632,50.4655916)\), there is not enough evidence to
  suggest the average pool temperature is different from the recommended
  \(50\) degrees.
\end{enumerate}

\begin{Shaded}
\begin{Highlighting}[]
\NormalTok{  Yst <-}\StringTok{ }\NormalTok{(}\DecValTok{1}\OperatorTok{/}\KeywordTok{length}\NormalTok{(AllTemp)) }\OperatorTok{*}\NormalTok{( (}\KeywordTok{length}\NormalTok{(Abv)}\OperatorTok{*}\KeywordTok{mean}\NormalTok{(Abv)) }\OperatorTok{+}\StringTok{  }\NormalTok{(}\KeywordTok{length}\NormalTok{(Bel)}\OperatorTok{*}\KeywordTok{mean}\NormalTok{(Bel)));}
\NormalTok{fpcAbv <-}\StringTok{ }\DecValTok{1} \OperatorTok{-}\StringTok{ }\NormalTok{(}\KeywordTok{length}\NormalTok{(Abv) }\OperatorTok{/}\KeywordTok{length}\NormalTok{(AllTemp) )}
\NormalTok{fpcBel <-}\DecValTok{1} \OperatorTok{-}\StringTok{ }\NormalTok{(}\KeywordTok{length}\NormalTok{(Bel) }\OperatorTok{/}\KeywordTok{length}\NormalTok{(AllTemp) )}

\NormalTok{Var <-}\StringTok{ }\NormalTok{((}\DecValTok{1}\OperatorTok{/}\KeywordTok{length}\NormalTok{(AllTemp))}\OperatorTok{^}\DecValTok{2}\NormalTok{)}\OperatorTok{*}\NormalTok{(}\KeywordTok{length}\NormalTok{(Bel))}\OperatorTok{^}\DecValTok{2}\OperatorTok{*}\NormalTok{fpcBel}\OperatorTok{*}\KeywordTok{sd}\NormalTok{(Bel)}\OperatorTok{^}\DecValTok{2}\OperatorTok{/}\KeywordTok{length}\NormalTok{(Bel)}
\NormalTok{Bd <-}\StringTok{ }\DecValTok{2}\OperatorTok{*}\KeywordTok{sqrt}\NormalTok{(Var)}
\end{Highlighting}
\end{Shaded}

\begin{enumerate}
\def\labelenumi{\alph{enumi}.}
\setcounter{enumi}{1}
\item
  We will now consider a stratified sample.
\item
  Treating the sample as stratified does improve our
\end{enumerate}

\end{document}
